%%%%%%%%%%%%%%%%%%%%%%%%%%%%%%%%%%%%%%%%%%%%%%%%%%%%%%%%%%%%%%%%%%%%%%
%% PREAMBLE
% \documentclass[handout,t]{beamer}            % HANDOUT
% \documentclass[handout,notes=show,t]{beamer} % NOTES
\documentclass[t]{beamer}       % SLIDES
\usetheme{FAU}                  % FAU format
\usepackage{beamer-tools}       % standard packages, definitions, macros

\usepackage{bibentry}           % for single bibentries
\nobibliography{}               % for showing bibentries outside references
\usepackage{hyperref}           % hyperlinks (import last)

% individual stuff
\DeclareMathOperator*{\argmax}{arg\,max}
\newcommand{\Prob}[1]{\mathbb{P}\left\{#1\right\}}
\newcommand{\Probest}[1]{\hat{\mathbb{P}}\left\{#1\right\}}

%%%%%%%%%%%%%%%%%%%%%%%%%%%%%%%%%%%%%%%%%%%%%%%%%%%%%%%%%%%%%%%%%%%%%%
%% Configuration of Titlepage

\title{\textbf{Werkzeuge und Infrastrukturen}}
\subtitle{Wintersemester 2016/17}

\author[Ahmed Albawabiji]{Ahmed Albawabiji}

\institute[]{
  Professur für Korpuslinguistik\\
  Studiengang Linguistische Informatik\\
  Friedrich-Alexander-Universität Erlangen-Nürnberg\\
  \secondary{\url{ahmed.albawabiji@fau.de}}
}
\date{Erlangen, 20.12.2016}              % comment for today's date
\setbeamertemplate{navigation symbols}{} % remove navigation symbols


%%%%%%%%%%%%%%%%%%%%%%%%%%%%%%%%%%%%%%%%%%%%%%%%%%%%%%%%%%%%%%%%%%%%%%
%% Header

\begin{document}
\frame{\titlepage}              % create titlepage
\hideLogo                       % don't show the logo on following pages


%%%%%%%%%%%%%%%%%%%%%%%%%%%%%%%%%%%%%%%%%%%%%%%%%%%%%%%%%%%%%%%%%%%%%%
%% Table of Contents

\frame[plain]{\frametitle{Inhaltsverzeichnis}\tableofcontents}

%%%%%%%%%%%%%%%%%%%%%%%%%%%%%%%%%%%%%%%%%%%%%%%%%%%%%%%%%%%%%%%%%%%%%%
%% Actual Frames


\section{Einführung}
\subsection{Pipelines}
\begin{frame}{Abstakt:Was sind Pipelines?}

Es ist die Aktion von Serialisierung.\\
\bigskip
Eine Gegenaktion ist die Deserialisierung oder Parallelisation.\\
\bigskip

Es ist eine Methode, die Ausgabe eines Prozesses mit der Ausgabe eines anderen Prozesses zu verbinden.\\
\end{frame}


\begin{frame}{Motivation}
Früher sah es so aus:\\
%von hier
\$ \url{python get_some_data.py}\\
\$ \url{python clean_some_data.py}\\
\$ \url{python join_other_data.py}\\
\$ \url{python do_stuff_with_data.py}\\
\bigskip
funktioniert gut und sogar schnell!\\
\textbf{Problem beim Weiterleiten!}\\
Programmierer(B):\\
Es hat nicht funktioniert...!?\\
Programmierer (A):\\
Hast du den die Daten schon gefiltert, bevor du sie miteinanderverbinden!?\\
\end{frame}

\begin{frame}{Eine mögliche Lösung:}


\url{do_everything.py script:}\\
\url{if __name__== '__main__':}\\
\bigskip
\textbf{Darunter schreibe ich alle zum Laufen gewollten Methoden
Und dann führe ich alles mit einem Befehl aus!}\\
\bigskip
%$     get_some_data()$\\
%$     clean_some_data()$\\
%$     join_other_data()$\\
%$     do_stuff_with_data()$\\
$python do_everything.py$\\

\end{frame}






\begin{frame}{Warum Pipelines?}
\begin{itemize}
\item Unsere Daten leicht bearbeiten und die Fehler besser auflösen.
\item Unsere Arbeit ganz ordentlich organizieren.
\item Mehr Kollaborationsmoeglichkeit, besser Codeverständnis zwischen den Programmierern und/oder den verschiedenen Betriebsystemen oder Editors. (Das Abschicken ist nicht immer sicher! ).
\item Daten von irgendwo holen, etwas damit machen und dann schicken sie wo anderes ab.

\end{itemize}
\end{frame}



\begin{frame}{Was sind Pipelines?}

\begin{itemize}
\item Es ist eine Methode, die die Ausgabe eines Prozesses mit der Ausgabe eines anderen Prozesses miteinanderverbindet.
\item  Pipes sind alle wichtigen Schritte oder Prozesse zur Vorbereitung oder /und Verarbeitung unsere Daten.
\item Es ist die Umformen von prototype zur Produktion.
\item So wird die Ausgabe einesProgramms als die Eingabe eines anderen Programm verwendet um am Ende etwas komplizierteres draus herauszubringen.
\item Data pipelines sind überall.
\item Nützlich von data als Ereignis(Produktion) zu denken!
\item Wir schreiben Programme die zusammen ein grosseres Programm machen.
\end{itemize}
\end{frame}



\begin{frame}{Bestandteile einer Pipe:}
 
\begin{itemize}
\item Standard input(stin) :
Datei, von  der wir Information bekommen, um damit bearbeiten zu können.

\item Standard output(stout) :
Datei, von  der wir Information bekommen, um damit bearbeiten zu können.

\item Standard error(sterr)  :
!! OOPS! Etwas ging schief! Bitte schön diese Fehlermeldung als Ausgabe !
stdout(1) processes (A) verbinden wir mit dem stdin(0) Prozesses (B)
durch den Bar $|$     (ist ja trivial)\\
\textbf{Command(A) $|$ Command (B)}\\

Deswegen sind sie auch  FIFO gennant(first in first out) 
\end{itemize}
\end{frame}

\begin{frame}
\begin{figure}
\includegraphics[width=\textwidth]{pipe.png}
\end{figure}
\end{frame}




\begin{frame}{Arten von Pipes}
\begin{figure}
\includegraphics[width=\textwidth]{Bier-pipe.png}
\end{figure}
\end{frame}



\begin{frame}{Arten von Pipes}
\begin{figure}
\includegraphics[width=\textwidth]{bidierekt.png}
\end{figure}
\end{frame}





%\begin{frame}
%Beispiel(Pipelines) und vllt Arten von Pipelines erkaeren!
%\end{frame}
\section{Hauptteil}


%%%%%%LUIGI%%%%%%
\subsection{Luigi}


\begin{frame}
\begin{figure}
\includegraphics[width=\textwidth]{luigi.png}
\end{figure}
\end{frame}


\begin{frame}{kleine Motivation}
Führen wir zurück zu unserer Motivation von Pipelines:\\
Aber was wenn es Fehler dran gäben, was macht man!\\
Man könnte das mit (try : und except) lösen!\\
Danach werden  wir einen Weihnachtsbaum haben! Umständlich!\\
Da könnten uns Packages wie Luigi, Airflow, Pinball oder auch andere retten.\\
\url{https://github.com/pditommaso/awesome-pipeline}
\end{frame}


\begin{frame}{Was ist Luigi}
\begin{itemize}
\item Luigi ist ein python package, das dabei hilft kompezierte Pipelines zu bilden.
\item Wurde vom Spotify entwickelt, um ihnen thousande von Jobs, die sie täglich  in einer bestimmten Reihenfolge führen,  ermöglicht zu bewältigen.
\item Es ist der Klebstoff, der die Prozesse miteinander zusammenbringt.
\item Abhängigkeit ist durch input und Output identifiziert. D.h. TaskB ist von TaskA abhängig .. bedeutet, Output von TaskA ist das Input (requires)von TaskB.

\end{itemize}
\end{frame}

\begin{frame}{ Vorteile von Luigi}
\begin{itemize}
\item Definieren von einander abhängige Tasks, die lange Zeit zum Laufen bruachen.
\item Tasks verketten, automaten, mit den daraus möglichen entstandenen Fehlern umgehen, die Reihenfolge sehr einfach ändern. 
 Tasks könnten alles sein, Daten von Datenbanken holen oder zu den Datenbanken schicken, maschinale lernen oder was anderes.
\item weiterzumachen von dem letzten laufenden Schritt.(muss nicht nochmal von Anfang)\\
\item Workflow management, Visualization(running, penning, done Tasks!)(dafür braucht man \$ luigid).
Aber:\\
Ausführung ist von der Reihenfolge abhängig.\\
keine vorhandenen Reihenfolgen!\\
Visualizahtion könnte besser sein!\\

\end{itemize}
\begin{center}
\$pip install luigi
\end{center}
\end{frame}


\begin{frame}{Magie von Luigi}
\begin{figure}
\includegraphics[width=\textwidth]{fetch.png}
\end{figure}

\end{frame}


\begin{frame}{ Drei Hauptfunktionen:}
Task: Ein Teil der Bearbeitung! …. Erweiterung von der Klasse luigi.Task.\\
Target: Output eines Tasks..kann alles sein! Datei erstellen, modifizieren oder sostiges!\\
\begin{itemize}
\item def requires(self):\\
pass \# list of dependencies
\item def output(self):\\
pass \# task output
\item def run(self):\\
pass \# tasks logik.
\end{itemize}

Für ein Task sind die drei erforderlich … wir könnten aber unsere Kette  vergrößern.\\
Beispiele dazu..
\end{frame}
\begin{frame}
\begin{figure}
\includegraphics[width=\textwidth]{tasks.png}
\end{figure}
\end{frame}

%%%%%%%Airflow%%%%%
\subsection{Airflow}
\begin{frame}{Airflow}
\begin{itemize}
\item Funktioniert mit Python, my SQL, Bash und auch anderen.\\
\item Seit 2015.
\item DAGs anstatt Piplines.
Directed Acyclic Graph:
Eine Sammlung von den Tasks, die wir zum Laufen bringen wollen!
In einer Art und Weise, die die Relationen und Abhängigkeit zwichen denen spiegelt.\\
DAGs werden in einer Datei in Airflow's Datei, die $ DAG_FOLDER$ heisst.
\item Beim ersten Lauf erstellt das Programm  airflow.cfg und da kann man Airflow configuration edieren.
Beispiel Task A sammelt Daten , B analisiert , C schickt Emails.\\
\$pip install airflow\\
Strukturierter Baum:
\url{https://airflow.incubator.apache.org/ui.html}
\end{itemize}
\end{frame}


\begin{frame}{Ein DAG}
\begin{figure}
\includegraphics[width=\textwidth]{G.png}
\end{figure}
\end{frame}

%%%%%Pinball%%%%%%
\subsection{Pinball}


\begin{frame}
\begin{itemize}
\item Auch Python Package.
\item Modifizierter PunktWorkflow manager and scheduler.
\item Werden parent-job und child-Job benutzt, um die Abhängigkeit zu bilden.
\item Meldungen durch die E-Mail.
\$ pip install pinball
\end{itemize}
\end{frame}






\section{Zusammenfassung und Vergleich}
\begin{frame}

Luigi Vs airflow Vs pinball\\
\bigskip
\url{https://www.michaelcho.me/article/data-pipelines-airflow-vs-pinball-vs-luigi}\\

%$http://bytepawn.com/luigi-airflow-pinball.html#luigi-airflow-pinball$\\

\end{frame}



%%%%%%Literaturverzeichnis%%%%%%%%%%%%%%%%%%%%

\section{Literaturverzeichnis}
\begin{frame}
Wachsmuth,Henning: Text Analyse und Pipelines,Springer,2015.\\
\url{https://github.com/pditommaso/awesome-pipeline}\\
\url{http://bionics.it/posts/luigi-tutorial}\\
\url{https://www.michaelcho.me/article/data-pipelines-airflow-vs-pinball-vs-luigi}\\
%\url{http://bytepawn.com/luigi-airflow-pinball.html#luigi-airflow-pinball}\\
\url{http://www.python-kurs.eu/pipes.php}
\end{frame}


\end{document}
